\documentclass[10pt]{beamer}
\usepackage[utf8]{inputenc}
\usepackage{adjustbox}
\usepackage{graphicx}
\setbeamertemplate{caption}[numbered] % enúmera las figuras


\usetheme[
%%% option passed to the outer theme
%    progressstyle=fixedCircCnt,   % fixedCircCnt, movingCircCnt (moving is deault)
  ]{Feather}
  
% If you want to change the colors of the various elements in the theme, edit and uncomment the following lines


% Change the bar colors:
%\setbeamercolor{Feather}{fg=red!20,bg=red}

% Change the color of the structural elements:
%\setbeamercolor{structure}{fg=red}

% Change the frame title text color:
%\setbeamercolor{frametitle}{fg=blue}

% Change the normal text color background:
%\setbeamercolor{normal text}{fg=black,bg=gray!10}

%-------------------------------------------------------
% INCLUDE PACKAGES
%-------------------------------------------------------

\usepackage[utf8]{inputenc}
\usepackage[english]{babel}
\usepackage[T1]{fontenc}
\usepackage{helvet}

%-------------------------------------------------------
% DEFFINING AND REDEFINING COMMANDS
%----------------------------------------------------
\newcommand{\mat}[1] {\mathbf{#1}}
\newcommand{\vect}[1]{\mathbf{#1}}

\renewcommand{\figurename}{Fig.}
% colored hyperlinks
\newcommand{\chref}[2]{
  \href{#1}{{\usebeamercolor[bg]{Feather}#2}}
}

%-------------------------------------------------------
% INFORMATION IN THE TITLE PAGE
%-------------------------------------------------------

\title[] % [] is optional - is placed on the bottom of the sidebar on every slide
{ % is placed on the title page
      \textbf{Procesamiento de señales de electroencefalograma usando redes neuronales para aplicaciones en sistemas BCI}
}

\subtitle[Procesamiento de señales de electroencefalograma usando redes neuronales para aplicaciones en sistemas BCI]
{
      \textbf{}
}

\author[C.J. Ortiz Echeverri]
{   Presenta:\\   
    MSc. César Javier Ortiz Echeverri  \\
    Director:\\
    Dr. Juvenal Rodríguez Reséndiz\\
    Codirector:\\
    Dr. Roberto Augusto Gómez Loenzo
      {}
}

\institute[]
{Universidad Autónoma de Querétaro, Facultad de Informática \\
  
  %there must be an empty line above this line - otherwise some unwanted space is added between the university and the country (I do not know why;( )
}

\date{\today}

%-------------------------------------------------------
% THE BODY OF THE PRESENTATION
%-------------------------------------------------------

\begin{document}

%-------------------------------------------------------
% THE TITLEPAGE
%-------------------------------------------------------

{\1% % this is the name of the PDF file for the background
\begin{frame}[plain,noframenumbering] % the plain option removes the header from the title page, noframenumbering removes the numbering of this frame only
  \titlepage % call the title page information from above
\end{frame}}


\begin{frame}{Contenido}{}
\tableofcontents
\end{frame}

% --------------------------------------------------------------------------------------------------------------
\section{Introducción}
\begin{frame}{Introducción}
Interfaz Cerebro-Computadora (BCI por sus siglas en inglés).
\begin{figure}[ht]
\centering
\includegraphics[scale=1.2]{Figures/introduccion}
\caption{Interfaz Cerebro-Computadora.}
\label{introduccion}
\end{figure}
\end{frame}

\begin{frame}{Introducción}
\begin{figure}[ht]
\centering
\includegraphics[scale=1.1]{Figures/BCI_etapas}
\caption{Etapas principales en un BCI.}
\label{BCI}
\end{figure}  
\end{frame}





\subsection{Motivación}
\begin{frame}{Motivación}
\begin{figure}[ht]
\centering
\includegraphics[scale=0.6]{Figures/motivacion.pdf}
\caption{Datos del INEGI (2010).}
\label{inegi}
\end{figure}

Según el Instituto Nacional de Geografía y Estadística (INEGI), al año 2010, en México, las personas que tienen algún tipo de discapacidad son 5 millones 739 mil 270, lo que representa 5.1\% de la población total.

\end{frame}

\subsection{Objetivos}
\begin{frame}{Objetivos}

Estudiar diferentes configuraciones de algoritmos de separación de fuentes ciegas y redes neuronales en implementaciones de BCI.

\begin{itemize}
\item Estudiar las características estadísticas de las señales MI-EEG.
\item Estudiar los fundamentos matemáticos de los algoritmos de BSS.
\item Diseñar una metodología para analizar el comportamiento de los algoritmos propuestos.
\item Implementar la etapa de extracción de características basadas en descriptores en el dominio del tiempo y la frecuencia.
\item Implementar etapas de clasificación basasdas en estrategias de \textit{Machine learning}.
\end{itemize}
\end{frame}




%-------------------------------------------------------

\section{Estado del arte}

\begin{frame}{Estado del arte}

\begin{table}[h]
\centering
\caption{Publicaciones relacionadas con BCI-MI}
\label{MI1}
\resizebox{\textwidth}{!}{ 
\begin{tabular}{lllll}
\hline
Autor           & Año & Método                                   & Clasificador   \\                         
\hline

1. Thomas~et~al. & 2009 & Banco de filtros-Patrones espaciales   & Máquinas de soporte vectorial          \\
2. Chin et~al.   & 2009 & Banco de filtros-Patrones espaciales   & Bayesianos             \\
3. Lu~et~al.     & 2010 & Patrones espaciales regularizados      & Análisis de discriminantes Fisher           \\
4. Wang~et al.   & 2012 & Análisis de componentes independientes & Análisis de discriminantes Fisher         \\
5. Zhang~et~al.  & 2013 &                                        & Análisis de discriminantes lineales           \\
6. Park~et~al.   & 2013 & Descomposición empírica                & Máquinas de soporte vectorial          \\
7. Aghaei~et~al. & 2016 & Patrones espaciales                    & Bayesianos              \\ 
8. Siuly~et~al.  & 2016 & Localización óptima                    & Bayesianos                           \\
9. Kervic~et al. & 2017 & Análisis de componentes principales    & k-vecino cercano                     \\         
10.Taran~et~al.  & 2018 & transformada wavelet ajustable         & Máquinas de soporte vectorial   \\
11.Taran~et~al.  & 2018 & Descomposición empírica                & Máquinas de soporte vectorial   \\
12.Islam~et~al.  & 2018 & Mapeo tangente espacial                & Máquinas de soporte vectorial          \\
13.Luo~et~al.    & 2018 & Descriptores espacio-temporales        & Máquinas de soporte vectorial          \\ 
14.Zhoua~et~al.  & 2019 & Análsis de componentes independientes  & análsis trial a trial                  \\
15.Guan~et~al.   & 2019 & Árboles de decisión                    & k-vecino cercano                       \\
\hline
\end{tabular}}
\end{table}
\end{frame}

\begin{frame}{Estado del arte}


\begin{table}[h]
\centering
\caption{\label{tabla 2}BCI-MI usando arquitecturas de deep learning.}
\label{DeepL}
\resizebox{\textwidth}{!}{ 
\begin{tabular}{lllll}
\hline
Autor                & Año & Forma de los datos de entrada      & Arquitectura       \\                         
\hline
1. X. An~et~al.       & 2014 & Series de tiempo                 & Red de creencias profundas\\
2. G. Xu~et~al.       & 2016 & mapas tiempo-frecuencia STFT     & Red convolucional pre-entrenada \\
3. Z. Tayeb~et~al.    & 2016 & mapas tiempo-frecuencia STFT     & Red convolucional y recurrente \\
4. Z. Yin, J. Zhang   & 2017 & Densidad espectral               & Autodecodificador de supresión de ruido apilado \\
5. N. Lu~et~al.       & 2017 & Espectros de frecuencia          & Máquina de Boltzmann restringida \\
6. Y. Tabar~et~al.    & 2017 & mapas tiempo-frecuencia STFT     & Red convolucional \\
7. Z. Tang~et~al.     & 2017 & Series de tiempo                 & Red convolucional \\
8. Z. Jiao~et~al.     & 2018 & mapas tiempo-frecuencia STFT     & Red convolucional \\
9. S.Chaudhary~et~al. & 2018 & mapas tiempo-frecuencia STFT/CWT & Red convolucional profunda (\emph{ALexNet}) \\
10.Ruffini~et~al.     & 2018 & mapas tiempo-frecuencia STFT     & Red convolucional y recurrente \\
11. Z. Zhang~et~al.   & 2019 & mapas tiempo-frecuencia CWT      & Red convolucional\\
12.M.Dai~et~al.       & 2019 & mapas tiempo-frecuencia STFT     & Red convolucional y  autodecodificador variacional \\

\hline
\end{tabular}}
\end{table}

 
\end{frame}



%-------------------------------------------------------

\section{Separación de Fuentes Ciegas}

\begin{frame}{Separación de Fuentes Ciegas}
\begin{figure}[h]
\centering
\includegraphics[scale=0.65]{Figures/BSS}
\caption{Mezcla de fuentes.}
\label{BSS}
\end{figure}
\end{frame}

\begin{frame}{Separación de Fuentes Ciegas}
Separación de Fuentes Ciegas (BSS por sus siglas en inglés)
\begin{equation}
\begin{split}
x_{1}(t) &= a_{11}s_{1}+a_{12}s_{2}, \\
x_{2}(t) &= a_{21}s_{1}+a_{22}s_{2}
\end{split}
\end{equation}  

Expresado en forma matricial 
\begin{equation}
\vect{x}(t) = \mat{A}\vect{s}(t)
\end{equation}

\noindent Donde $\vect{x}(t)$ es el vector de las señales mezcladas, $\mat{A}$ es la matriz de mezcla. BSS consiste en encontrar una matriz $\mat{B}$ que en un caso ideal $\mat{B}=\mat{A}^{-1}$ 

\begin{equation}
\vect{s}(t) = \mat{B}\vect{x}(t)
\end{equation}

\end{frame}

\begin{frame}{Separación de Fuentes Ciegas}
\begin{figure}[h]
\centering
\includegraphics[scale=0.4]{Figures/BSS1}
\caption{Blanqueamiento y rotación.}
\label{BSS1}
\end{figure}

\end{frame}



\begin{frame}{Separación de Fuentes Ciegas}
\begin{figure}[ht]
\centering
\includegraphics[scale=0.6]{Figures/whiteningMatrix}
\caption{Matriz de covarianza (a) original (b) blanqueada.}
\label{whiteningMatrix}
\end{figure}

\begin{equation}
\vect{z(t)} = \mat{W}\vect{x(t)}
\end{equation}


\begin{equation}
\mat{R_z}= \mathbb{E}\left(\vect{z}\vect{z}^T\right)=\mat{I}
\end{equation}



\end{frame}

\begin{frame}{Separación de Fuentes Ciegas}
\begin{figure}[ht]
\centering
\includegraphics[scale=0.4]{Figures/whiteExmaple.pdf}
\caption{Blanqueamiento para dos señales (a) fuentes originales (b) distribución de las fuentes originales (c) fuentes mezcladas (d) distribución fuentes mezcladas (e) fuentes blanqueadas (f) distribución fuentes blanqueadas. }
\label{whitening}
\end{figure}
\end{frame}






\begin{frame}{Separación de Fuentes Ciegas}

\begin{equation}
\vect{s}(t) = \mat{B}\vect{x}(t)
\end{equation}

\begin{equation}
\vect{z(t)} = \mat{W}\vect{x(t)}
\end{equation}

\noindent se puede definir una matriz de correlación $\mat{R}_{x(\tau)}$ expresada como

\begin{equation}
\mat{R}_{x(\tau)}= \mat{A}\mat{R}_{s(\tau)}\mat{A}^T
\end{equation}
  
\begin{equation}
\mat{R}_{z(\tau)}=\mat{W}\mat{R}_{x(\tau)}\mat{W}^T=\mat{J} {\Sigma_2} \mat{J}^T
\end{equation}


\noindent En donde ${\Sigma_2}$ es una matriz diagonal y $\mat{J}$ es una matriz ortogonal. 



\begin{equation}
\mat{B} = \mat{J}^T\mat{W}
\end{equation} 

\end{frame}


\begin{frame}{Separación de Fuentes Ciegas}

\begin{figure}[ht]
\centering
\includegraphics[scale=0.6]{Figures/SOBI}
\caption{Algoritmo SOBI basado en diagonalización de aproximación conjunta de eigen-matrices.}
\label{SOBI}
\end{figure}
\end{frame}

\begin{frame}{Separación de Fuentes Ciegas}

\begin{figure}[ht]
\centering
\includegraphics[scale=0.6]{Figures/ICA}
\caption{Algoritmo ICA basado en la distribución de densidad de probabilidad.}
\label{ICA}
\end{figure}
\end{frame}

\begin{frame}{Separación de Fuentes Ciegas}


$p_1(s_1)$ y $p_2(s_2)$ dos densidades de probabilidad de dos componentes, $\vect{s}$.
Kullback-Leibler mide la diferencia entre la densidad y el producto de las densidades marginales.

\begin{equation}
    D(p_1|p_2)=\int_{-\infty}^{\infty}p_1(s_1)\log\left(\frac{p_1(s_1)}{p_2(s_2)}\right) 
\end{equation} 

Las variables aleatorias se consideran estadísticamente independientes si y sólo si
\begin{equation}
p(\vect{s})=p_{s}(s_{1},s_{2},...,s_{n})=p_{1}(s_{1})p_{2}(s_{2})...p_{n}(s_{n})=\prod_i{p_i(s_i)}
\end{equation}
    
    
\end{frame}




\begin{frame}{Separación de Fuentes Ciegas}
\begin{figure}[h]
\centering
\includegraphics[scale=0.4]{Figures/rotationExample.pdf}
\caption{Rotación para dos señales (a) fuentes originales (b) distribución de las fuentes originales (c) fuentes mezcladas (d) distribución fuentes mezcladas (e) fuentes blanqueadas (f) distribución fuentes blanqueadas. }
\label{SOS}
\end{figure}
\end{frame}

\section{Metodología}

\begin{frame}{Metodología}
\begin{figure}[ht]
\centering
\includegraphics[scale=0.65]{Figures/sortedFFT.pdf}
\caption{Criterio de ordenamiento de BSS.}
\label{sortedfft}
\end{figure}
\end{frame}


\begin{frame}{Experimentación}

\begin{figure}[ht]
\centering
\includegraphics[scale=0.6]{Figures/dataset.pdf}
\caption{Base de datos usada.}
\label{SOBI}
\end{figure}
\end{frame}





%-------------------------------------------------------
\section{Extracción y clasificación de características}

\begin{frame}{Extracción y clasificación de características}

\begin{table}[h]
\caption{descriptores en el dominio del tiempo}
\centering
\resizebox{0.5\textwidth}{!}{
\label{T1}  
\begin{tabular}{lll}
\hline
Descriptor              & Ecuación                              \\
\hline
1. Cruce por cero    &   $Z(i)=\frac{1}{2L}\sum_i^L|sgn[x_i(n)]-sgn[x_i(n-1)]|^2$\\\\

2. Energía            &   $E(i)=\frac{1}{L}\sum_i^L|x_i(n)|^2 $ \\\\
                    
3. Entropía &   $H(i)=-\sum_{j=1}^k{e_jlog_2(e_j)}$  \\\\
\hline
\end{tabular}}

\begin{table}[h]
\centering
\caption{descriptores en el dominio de la frecuencia}
\resizebox{0.5\textwidth}{!}{
\label{T2}  
\begin{tabular}{ll}
\hline
Descriptor              & Ecuación                              \\
\hline
4. Centroide espectral &  $C_i=\frac{\sum_1^{W_{fl}}kX_i(k) }{\sum_1^{W_{fl}}X_i(k)}$\\\\
5. Dispersión espectral   &  $S_i=\sqrt{\frac{\sum_1^{W_{fl}}(k-C_i)^2X_i(k) }{\sum_1^{W_{fl}}X_i(k)}}$  \\\\
6. Entropía espectral  &  $H(i)=-\sum_{f=0}^{L-1}{n_flog_2(n_f)}$          \\\\
7. Flujo espectral     &  $Fl_{(i,i-1)}=-\sum_{f=1}^{Wf_L}{(EN_i(k)-EN_{i-1}(k))^2}$ \\\\
8. Desplazamiento espectral  &  $\sum_{k=1}^m{X_i(k)}=C\sum_{k=1}^{Wf_L}{X_i(k)}$ \\\\
9. Coeficientes Cepstrales Frecuencia Mel        &   $c_m=\sum_{k=1}^{L}{(log(O_k))cos\left[m\left(k-\frac{1}{2}\right)\frac{\pi}{L}\right]}$\\\\
\hline
\end{tabular}}
\end{table}

\end{table}
\end{frame}


\begin{frame}{Extracción y clasificación de características}
\begin{figure}[h]
\centering
\includegraphics[scale=0.65]{Figures/MFCC}
\caption{Coeficientes Cepstrales Frecuencia Mel (MFCC).}
\end{figure}
\end{frame}






\begin{frame}{Extracción y clasificación de características}

\begin{figure}[h]
\centering
\includegraphics[scale=0.35]{Figures/clasificacion.eps}
\caption{mapa de descriptores y MLP.}
\label{clasificacion}
\end{figure}

\begin{itemize}
    \item Grupo 1: dominio del tiempo.
    \item Grupo 2: dominio de la frecuencia.
    \item Grupo 3: MFCC.
\end{itemize}




\begin{table}[H]
\caption{porcentaje de clasificación sin aplicar BSS.}
\centering
\resizebox{0.5\textwidth}{!}{
\label{noBSS} % Give a unique label
\begin{tabular}{lllll|c}
\hline\noalign{\smallskip}
Sujeto    & Grupo 1 & Grupo 2  & Grupo 3      & Todos  & Tiempo por cada trial (s) \\
\noalign{\smallskip}\hline\noalign{\smallskip}
IVa-aa     & 52.5    & 50.7     & 40.6         & 53.0 & 0.1190\\
IVa-al     & 50.6    & 66.7     & 38.7         & 47.6 & 0.1200\\
IVa-av     & 62.2    & 21.7     & \textbf{91.5}& 72.5 & 0.1222\\
IVa-aw     & 50.5    & 45.4     & 35.0         & 44.2 & 0.1194\\
IVa-ay     & 52.2    & 26.3     & 61.3         & 23.3 & 0.1219\\
\hline\noalign{\smallskip}
media       & 53.6    & 42.1     & 53.4         & 48.1 & 0.1204 \\
des. std        & 3.9     & 15.3     & 19.3         & 14.4 & 0.0015 \\
\noalign{\smallskip}\hline
\end{tabular}}
\end{table}

\end{frame}




\begin{frame}{Extracción y clasificación de características}
    

\begin{table}[H]
\caption{Porcentaje de clasificación usando SOBI}
\centering
\resizebox{0.5\textwidth}{!}{
\label{SOBI1} % Give a unique label
\begin{tabular}{lllll|c}
\hline\noalign{\smallskip}
Sujeto    & Grupo 1 & Grupo 2  & Grupo 3      & Todos  & Tiempo por cada trial (s) \\
\noalign{\smallskip}\hline\noalign{\smallskip}
IVa-aa     & 67.0    & 82.3    & 65.9   & 78.4 & 0.2079\\
IVa-al     & 89.6    & 78.5    & 61.5   & 77.3 & 0.1607\\
IVa-av     & 55.9    & 66.7    & 70.1   & 85.9 & 0.1621\\
IVa-aw     & 72.1    & 83.2    & 75.3   & 81.3 & 0.1614\\
IVa-ay     & 56.4    & 80.7    & 72.5   & 78.5 & 0.1600
\\
\hline\noalign{\smallskip}
media    & 68.2   & \textbf{78.3}     & 69.0   & \textbf{78.5} & 0.1704 \\
des. std & 13.8   & 6.7      & 5.4    & 6.0  & 0.0209\\
\noalign{\smallskip}\hline
\end{tabular}}
\end{table}

\begin{table}[H]
\caption{Porcentaje de clasificación usando ICA}
\centering
\resizebox{0.5\textwidth}{!}{
\label{ICA1} % Give a unique label
\begin{tabular}{lllll|c}
\hline\noalign{\smallskip}
Sujeto    & Grupo 1 & Grupo 2  & Grupo 3      & Todos  & Tiempo por cada trial (s) \\
\noalign{\smallskip}\hline\noalign{\smallskip}
IVa-aa     & 78.5    & 90.3    & 76.0    & 91.1 & 1.4750\\
IVa-al     & 78.8    & 88.7    & 96.5    & 90.7 & 1.4464\\
IVa-av     & 78.1    & 87.6    & 92.2    & 91.0 & 1.4536\\
IVa-aw     & 88.3    & 89.2    & 92.8    & 92.3 & 1.4464\\
IVa-ay     & 89.4    & 90.1    & 90.4    & 89.0 & 1.4500\\
\\
\hline\noalign{\smallskip}
media    & 82.6    & 89.2    & \textbf{89.6} & \textbf{90.8} & 1.4543\\
des. std        & 5.7    & 1.0      & 7.9     & 1.2  & 0.0120\\
\noalign{\smallskip}\hline
\end{tabular}}
\end{table}


\end{frame}


\begin{frame}{Extracción y clasificación de características}

\begin{equation}
W_x(a,\tau)=\frac{1}{\sqrt{|a|}} \int_{-\infty}^{\infty} x(t)\psi^{\ast}\left(\frac{t-\tau}{a}\right)\,dt,
\end{equation}

\begin{figure}[ht]
\centering
\includegraphics[scale=0.5]{Figures/morletCWT.pdf}
\caption{(a) Onda senoidal (b) señal de EEG.}
\label{CWT1}
\end{figure}
    
\end{frame}



\begin{frame}{Extracción y clasificación de características}
\begin{figure}[ht]
\centering
\includegraphics[scale=0.6]{Figures/CNN_esquema.pdf}
\caption{Esquema de una Red Neuronal Convolucional.}
\label{CNN}
\end{figure}
\end{frame}



\begin{frame}{Extracción y clasificación de características}

\begin{table}[h!]
\caption{Arquitectura propuesta}
\centering
\resizebox{0.7\textwidth}{!}{
\renewcommand{\arraystretch}{1.2}

\begin{tabular}{c|l|cc|l}
\hline
Capa & Operación                      & Kernel  & Stride  &  Salida\\ 
\hline
1     & Conv2D $\rightarrow 250$       & $(3,1)$ & $(2,2)$ &  (63,256,250)\\
      & Activation$\rightarrow ReLU$   &         &         &  (63,256,250)\\
      & Max-pooling$\rightarrow(4,4)$  &         &         &  (15,64,250)\\
\hline
2     & Conv2D $\rightarrow 150$       & $(1,2)$ & $(1,1)$ &  (15,63,150)\\
      & Activación$\rightarrow ReLU$   &         &         &  (15,63,150)\\
      & Max-pooling$\rightarrow(3,3)$  &         &         &  (5,21,150)\\
\hline
3     & Aplanado                        &         &         &  (15750)\\
      & Densa$\rightarrow2048$         &         &         &  (2048)\\
      & Activación$\rightarrow ReLU$   &         &         &  (2048)\\
      & Dropout$\rightarrow 0.4$       &         &         &  (2048)\\      
\hline
4     & Dense$\rightarrow2$            &         &         &  (2)\\
      & Activación$\rightarrow softmax$&         &         &  (2)\\
\hline
\end{tabular}}
\label{architecture}
\end{table}
    
\end{frame}




\begin{frame}{Extracción y clasificación de características}

\begin{figure}[h]
\centering
\includegraphics[scale=0.7]{Figures/methodCNN}
\caption{Metodología propuesta usando CNN.}
\label{methodCNN}
\end{figure}

\begin{figure}[h]
\centering
\includegraphics[scale=0.27]{Figures/CWT}
\caption{Mapas de CWT (a) Una fuente, (b) Todas las fuentes.}
\label{CWT}
\end{figure}

\end{frame}

\begin{frame}{Red Neuronal Convolucional (CNN)}

\begin{figure}[h]
\centering
\includegraphics[scale=0.75]{Figures/train_test}
\caption{(a) sin BSS, (b) ICA no ordenado, (c) SOBI ordenado, (d) ICA ordenado (30 epochs).}
\label{train_test}
\end{figure}
\end{frame}





\begin{frame}{Red Neuronal Convolucional (CNN)}

\begin{table}[h]
\caption{clasificación 10-validación cruzada }
\resizebox{0.8\textwidth}{!}{%
\centering
\renewcommand{\arraystretch}{1.2}
\begin{tabular}{l|cccccccccc|c|c}
\textbf{Sujeto}    & & & & &\textbf{Clasificación}\\
\hline
       & k=1  & k=2  & k=3  & k=4  & k=5  & k=6  & k=7  & k=8  & k=9  & k=10 &Promedio & des. std\\ 
\hline
aa &94.79&100.00&96.87&89.58&100.00&100.00&100.00 &98.95 &98.95&98.95 &97.81 & 3.34\\
al &91.66 &94.79 &94.79 &98.95 &85.41 &87.50 &97.91 &100.00&98.95 &94.79 &94.47 & 4.96\\
av &95.83 &97.91 &100.00&98.95 &97.91 &88.54 &68.75 &100.00&100.00&100.00&94.78 & 9.79\\
aw &98.75 &92.50 &95.00 &99.37 &99.37 &91.87 &100.00&98.75 &76.87 &88.12 &94.06 & 7.26\\
ay &85.41 &97.91 &85.41 &79.16 &96.87 &95.83 &95.83 &100.00 &96.87&88.54 &92.18 & 6.98\\
\hline
Promedio    &      &      &      &      &      &      &      &       &     &      &\textbf{94.66} & 6.46\\
        \hline
    \end{tabular}}
    
    \label{kfoldcross1}
\end{table}


\begin{table}[h]
\caption{Comparación con otros trabajos reportados que usan la misma base de datos.}
\resizebox{0.5\textwidth}{!}{%
\centering
\renewcommand{\arraystretch}{1.2}
\begin{tabular}{lllcc}
\textbf{Autor}& \textbf{Método}& \textbf{Clasificador} & \textbf{Clasificación} (\%) & \textbf{Año}\\
\hline
Lu et al.    & R-CSP                 & R-CSP       & 83.90         & 2010\\
Siuly et al. & CT                     & LS-SVM     & 88.32         & 2011\\
Zhang et al. & Z-score                & LDA        & 81.10         & 2013\\
Siuly et al. & OA                     & NB         & 96.36         & 2016\\
Kevric et al.& MSPCA, WPD, HOS        & k-NN       & 92.80         & 2017\\
Taran et al. & TQWT                   & LS-SVM     & 96.89         & 2018\\
Propuesto     & fastICA Ordenado-CWT     & CNN        & \textbf{94.21}& 2019\\
\hline
\end{tabular}}
\label{works}
\end{table}
 
 
 
    
\end{frame}






\section{Análisis de resultados}


\section{Conclusiones}

\begin{frame}{Conclusiones}

\begin{itemize}

    \item Uso de BSS para mejorar resolución espacial ordenado de acuerdo a características frecuenciales de las señales de MI-EEG.
    \item Representaciones tensoriales tiempo-frecuencia-espacio son adecuadas para la búsqueda de características relevantes en las señales EEG. 
    \item Contribución en el estudio de arquitecturas usadas en \textit{deep learning} en señales EEG.
    \item MFCC como posibles descriptores para EEG.
\end{itemize}
    
\end{frame}


\section{Trabajos futuros}

\begin{frame}{Trabajos futuros}

\begin{itemize}
    
    \item Buscar nuevas representaciones tensoriales de las señales de entrada para las CNN.
    \item Estudiar las zonas de mayor influencia en las imágenes de CWT para mejorar el diseño en la CNN.
    \item Usar aumento de datos (\textit{data augmentation}) para buscar una mejor generalización en el aprendizaje. 
    \item Probar una arquitectura de \textit{deep learning} end-to-end sin usar BSS.
    \item Ajustar los parámetros del algoritmo SOBI para usar SOS en lugar de HOS.
    \item Ajustar los coeficientes MFCC para señales de EEG.
    \item Implementar algoritmos en sistemas embebidos para aplicaciones reales de BCI.
\end{itemize}
    
\end{frame}

\begin{frame}{Artículos indexados en el JCR realizados durante el doctorado}

\begin{enumerate}

\item C. J Ortiz-Echeverri, S. Salazar-Colores, J. Rodríguez-Reséndiz, and R. A. Gómez Loenzo (2019). \textbf{A New Approach for Motor Imagery Classification based on Sorted Blind Source Separation, Continuous Wavelet Transform, and Convolutional Neural
Network}. Sensors, 19(20):4541.


\item Ortiz‐Echeverri, C. J., Rodríguez‐Reséndiz, J., and Garduño‐Aparicio, M. (2018). \textbf{An approach to STFT and CWT learning through music hands‐on labs}. Computer Applications in Engineering Education, 26(6), 2026-2035.

 \item  Salazar-Colores, S., Ramos-Arreguín, J.-M., Ortiz Echeverri, C. J., Cabal-Yepez, E., Pedraza-Ortega, J.-C., and Rodríguez-Resendiz, J. (2018). \textbf{"Image dehazing using morphological opening, dilation and Gaussian filtering"}. Signal, Image and Video Processing. https://doi.org/10.1007/s11760-018-1286-9.
 
 \end{enumerate}
    
\end{frame}

\begin{frame}{Articulos presentados en congresos internacionales}
\begin{enumerate}
\item C. J. Ortiz-Echeverri, O. Paredees, S. Salazar-Colores, J. Rodríguez-Reséndiz, and R. Romo-Vázquez. \textbf{Comparative Study of time and Frequency features for EEG Classification.} In 2019 VIII Congreso Latinoamericano de Ingeniería Biomédica, Springer.

\item  C. J. Ortiz-Echeverri, K. D. Pérez , G. Galindo, and J. Rodríguez-Reséndiz (2018, May). \textbf{Blind source separation problem algorithms for audio and biomedical signals}. In 2018 XIV International Engineering Congress (CONIIN) (pp. 1-7). IEEE.


\item C. J. Ortiz-Echeverri, G. Galindo-Burgos, J.A. García-Hernández, G. Macías-Bobadilla, and J. Rodríguez-Reséndiz. \textbf{Music Classification using time-frequency descriptors and Multilayer Perceptron}. In 2019 XV International Engineering Congress (CONIIN) 
\end{enumerate}
    
\end{frame}

\begin{frame}{}

\begin{figure}[h]
\centering
\includegraphics[scale=0.75]{Figures/gracias.pdf}

\label{train_test}
\end{figure}

\end{frame}



%------------------------------------------------------





\end{document}