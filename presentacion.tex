\documentclass[10pt]{beamer}
\usepackage[utf8]{inputenc}
\usepackage{adjustbox}
\usepackage{graphicx}
\setbeamertemplate{caption}[numbered] % enúmera las figuras


\usetheme[
%%% option passed to the outer theme
%    progressstyle=fixedCircCnt,   % fixedCircCnt, movingCircCnt (moving is deault)
  ]{Feather}
  
% If you want to change the colors of the various elements in the theme, edit and uncomment the following lines


% Change the bar colors:
%\setbeamercolor{Feather}{fg=red!20,bg=red}

% Change the color of the structural elements:
%\setbeamercolor{structure}{fg=red}

% Change the frame title text color:
%\setbeamercolor{frametitle}{fg=blue}

% Change the normal text color background:
%\setbeamercolor{normal text}{fg=black,bg=gray!10}

%-------------------------------------------------------
% INCLUDE PACKAGES
%-------------------------------------------------------

\usepackage[utf8]{inputenc}
\usepackage[english]{babel}
\usepackage[T1]{fontenc}
\usepackage{helvet}
\usepackage{multirow}
\usepackage{todonotes}
\usepackage{makecell}
\usepackage{enumitem}
\usepackage{array}
\usepackage{amsmath}
\usepackage{tabularx}
\usepackage{bm,upgreek}
\usepackage{afterpage}
\usepackage{gensymb}
\usepackage{float}
\usepackage{marvosym}

%-------------------------------------------------------
% DEFFINING AND REDEFINING COMMANDS
%----------------------------------------------------
\newcommand{\mat}[1] {\mathbf{#1}}
\newcommand{\vect}[1]{\mathbf{#1}}

\renewcommand{\figurename}{Fig.}
% colored hyperlinks
\newcommand{\chref}[2]{
  \href{#1}{{\usebeamercolor[bg]{Feather}#2}}
}

%-------------------------------------------------------
% INFORMATION IN THE TITLE PAGE
%-------------------------------------------------------

\title[] % [] is optional - is placed on the bottom of the sidebar on every slide
{ % is placed on the title page
      \textbf{Desarrollo de un sistema de acondicionamiento de señal para determinar elementos en un sistema de clasificación para botellas PET post-consumo}
}

\subtitle[Desarrollo de un sistema de acondicionamiento de señal para determinar elementos en un sistema de clasificación para botellas PET post-consumo.]
{
      \textbf{}
}

\author[Javier Luna]
{   Presenta:\\   
    Javier Iván Luna Ramos  \\
    Director:\\
    M.I. Luis Alberto Ángeles Hurtado \\
    Codirector:\\
    Dr. Juvenal Rodríguez Reséndiz
      {}
}

\institute[]
{Universidad Autónoma de Querétaro, Facultad de Ingeniería \\
  
  %there must be an empty line above this line - otherwise some unwanted space is added between the university and the country (I do not know why;( )
}

\date{\today}

%-------------------------------------------------------
% THE BODY OF THE PRESENTATION
%-------------------------------------------------------

\begin{document}

%-------------------------------------------------------
% THE TITLEPAGE
%-------------------------------------------------------

{\1% % this is the name of the PDF file for the background
\begin{frame}[plain,noframenumbering] % the plain option removes the header from the title page, noframenumbering removes the numbering of this frame only
  \titlepage % call the title page information from above
\end{frame}}


\begin{frame}{Contenido}{}
\tableofcontents
\end{frame}

% --------------------------------------------------------------------------------------------------------------
\section{Introducción}

\begin{frame}{Introducción}
Los plásticos se definen como polímeros, en función de su estructura y su comportamiento, cuando son expuestos a la temperatura pueden clasificarse en termoplásticos, termofijos y elastómeros.
\begin{table}[h]
    \centering
    \caption{Símbolos para identificar el tipo de plástico reciclado \cite{Ramli2008}.}
    \label{Sımbolos}
    \begin{center}
    \begin{tabular}{ c c c c c c c }
        \hline
        \includegraphics[trim = {31mm 225mm 165mm 31mm},clip,scale=0.38]{img/Vg_fig01.pdf} & 
        \includegraphics[trim = {52mm 225mm 142mm 31mm},clip,scale=0.38]{img/Vg_fig01.pdf} & 
        \includegraphics[trim = {74mm 225mm 120mm 31mm},clip,scale=0.38]{img/Vg_fig01.pdf} &
        \includegraphics[trim = {96mm 225mm 98mm 31mm},clip,scale=0.38]{img/Vg_fig01.pdf} &
        \includegraphics[trim = {118mm 225mm 76mm 31mm},clip,scale=0.38]{img/Vg_fig01.pdf} &
        \includegraphics[trim = {140mm 225mm 54mm 31mm},clip,scale=0.38]{img/Vg_fig01.pdf} &
        \includegraphics[trim = {162mm 225mm 32mm 31mm},clip,scale=0.38]{img/Vg_fig01.pdf} \\
            ``PET'' & ``HDPE'' & ``PVC'' & ``LDPE'' & ``PP'' & ``PS'' & OTROS \\ 
            
    \end{tabular}
    \begin{tabular}{l c c c c c c}
        PET:\textit{Polyethylene Terephthalate} & & HDPE: \textit{High Density Polyethylene}\\
        PVC:\textit{Polyvinyl Chloride} & & LDPE: \textit{Low Density Polyethylene}\\
        PP:\textit{Polypropylene} & & PS: \textit{Polystyrene}\\
        \hline
    \end{tabular}
    \end{center}
    
\end{table}
\end{frame}

\begin{frame}{Introducción}
\begin{table}[h]
    \caption{Sistemas comerciales para la clasificación de diferentes tipos de plásticos reciclables \cite{Angeles2021}. }
    \label{Sistemas}
    \resizebox{10cm}{!} {
    \begin{tabular}{c m{33 em}}
         \hline
           Sistema & Ventajas \\  
         \hline
           1 & Considera el tamaño de la botella, remueve contaminantes, trabaja con 16 sensores de rayos infrarrojos, clasificación en 5 milisegundos, identifica color, utiliza un sistema neumático expulsando las botellas. \\
           2 & Utiliza una cámara de color, separa las botellas en orden individual, remueve contaminantes, velocidad de detección de 15 veces por segundo e identificación de 19 milisegundos, identifica tipos de colores en las botellas. \\
          
          3 & Alta detección al utilizar el reflejo de rayos x y de sensores infrarrojos por medio de espectroscopia este método es recomendado para detectar PCV. Analiza de 200 veces por segundo, separa 2 a 3 botellas por segundo. \\
          %&4 & & &  \\
         \hline
    \end{tabular}
    }
    \resizebox{10cm}{!} {
    \begin{tabular}{l}
            1 - \textit{BottleSort}, 2- \textit{Poly-Sort}, 3- \textit{Near infrared spectroscopic} y VS-2 \\
         \end{tabular}
    }
\end{table}
\end{frame}


\subsection{Motivación}
\begin{frame}{Motivación}
En México se generan más de 44 millones de toneladas anuales de residuos, y se espera que la cifra se incremente a 65 millones para el año 2030.
La mitad de nuestros residuos son orgánicos, y la otra mitad son sólidos, e incluyen plástico, papel, cartón, vidrio y aluminio.
El 90{$\%$} de estos desperdicios se arrojan a tiraderos de cielo abierto o a rellenos sanitarios lo que provoca contaminación del aire y el suelo.
Los plásticos reciclados son materiales muy utilizados a nivel global. 
En México de todos estos residuos que se producen solo el 5{$\%$} se recicla o valoriza lo que equivaldría a unas 2.2 millones de toneladas anuales \cite{SEMARNAT2019}, \cite{Angeles2021}.

\end{frame}

\subsection{Objetivos}
\begin{frame}{Objetivos}

Desarrollar e implementar una metodología que permita seleccionar los componentes significativos que contribuyen al correcto funcionamiento de un sistema de clasificación de “PET” post-consumo utilizando tecnologías previas en un periodo de un año y medio aproximadamente.

\begin{itemize}
    \item Diseñar y desarrollar un sistema de acondicionamiento de señal para el diodo de ``InGaAs''.
    \item Justificar a partir de una metodología los elementos eléctricos y electrónicos que interactuarán en el sistema.
    \item Determinar una metodología a partir de las características de los diodos ``InGaAs'' para seleccionar los elementos mecánicos.
    \item Comparar el comportamiento del diodo ``InGaAs'' con los distintos elementos electrónicos y eléctricos del sistema.
\end{itemize}
\end{frame}




%-------------------------------------------------------

\section{Estado del arte}

%-------------------------------------------------------

\section{Metodología}

\begin{frame}{Metodología}

\end{frame}


\begin{frame}{Experimentación}

\end{frame}





%-------------------------------------------------------

\section{Análisis de resultados}


\section{Conclusiones}

\begin{frame}{Conclusiones}

    
\end{frame}

\section{Bibliografía}

\begin{frame}{Referencias}

\bibliographystyle{unsrt}
\addcontentsline{toc}{section}{Referencias}
\bibliography{tesis.bib}

\end{frame}

\begin{frame}{Agradecimientos}

\begin{figure}[h]
\centering
\includegraphics[scale=0.75]{Figures/gracias.pdf}

\label{train_test}
\end{figure}

\end{frame}


%------------------------------------------------------





\end{document}